\documentclass{article} % For LaTeX2e
\usepackage{nips12submit_e,times}
\usepackage{color}
\usepackage{amssymb}
\usepackage{amsmath}
\usepackage{amsthm}

%\documentstyle[nips12submit_09,times,art10]{article} % For LaTeX 2.09

\title{Probabilistic Computation for Information Security}

\author{
Piotr Mardziel\\
Department of Computer Science\\
University of Maryland, College Park\\
College Park, MD 20740 \\
\texttt{piotrm@cs.umd.edu} \\
\And
Kasturi Rangan Raghavan\\
Department of Computer Science\\
University of California, Los Angeles\\
Los Angeles, CA 90024\\
\texttt{kr@cs.ucla.edu} \\
}

\newcommand{\fix}{\marginpar{FIX}}
\newcommand{\new}{\marginpar{NEW}}

\newcommand{\pxm}[1]{\textcolor{red}{PM -- #1}}
%\newcommand{\pxm}[1]{}

\nipsfinalcopy % Uncomment for camera-ready version

\newcommand{\ra}{\rightarrow}
\newcommand{\Real}{\mathbb{R}}

\newcommand{\blacklists}[0]{\textbf{Blacklists}}
\newcommand{\whitelists}[0]{\textbf{Whitelists}}

\newcommand{\secrets}[0]{\textbf{Secrets}}
\newcommand{\asecret}[0]{s}
\newcommand{\rsecret}[0]{\asecret^*}
%\newcommand{\rsecret}[0]{s}


\newcommand{\dists}[0]{\textbf{Dist}}
\newcommand{\outs}[0]{\textbf{Outs}}

\newcommand{\sconst}[1]{\ensuremath{\mathsf{#1}}}
\newcommand{\strue}{\sconst{True}}
\newcommand{\sfalse}{\sconst{False}}

\newcommand{\paren}[1]{\left( #1 \right)}
\newcommand{\set}[1]{\left\{ #1 \right\}}

\newcommand{\cond}[0]{|}
\newcommand{\acond}[0]{|^\sim}

\newcommand{\lsep}[0]{. \;}
\newcommand{\qsep}[0]{\; . \;}

\newcommand{\stacklabel}[1]{\stackrel{\smash{\scriptscriptstyle \mathrm{#1}}}}
\newcommand{\defeq}{\stacklabel{def}=}

\makeatletter
\newtheorem*{rep@theorem}{\rep@title}
\newcommand{\newreptheorem}[2]{%
\newenvironment{rep#1}[1]{%
 \def\rep@title{#2 \ref{##1}}%
 \begin{rep@theorem}}%
 {\end{rep@theorem}}}
\makeatother
\theoremstyle{plain} % bold head, italicized body  (the default)
\newtheorem{theorem}{Theorem}
\newreptheorem{theorem}{Theorem}
\theoremstyle{definition} % bold head, normal body
\newtheorem{definition}[theorem]{Definition}
\newtheorem*{definition-un}{Definition}

\begin{document}

\maketitle

Probabilistic computation is a convenient means of mechanically
reasoning about a variety of information security problems. At its
core, information security concerns itself with measuring or limiting
the knowledge an adversary might attain from interacting with a
protected system. Probabilistic inference lets one compute this
knowledge explicitly as long as the interaction can be described as a
program in a suitable probabilistic language that supports
conditioning. Security concerns, however, require soundness guarantees
on probabilistic inference which are generally not present in machine
learning applications. We summarize some recent work on probabilistic
computing for information security and highlight challenging aspects
that still need to be addressed.

\paragraph*{Tracking knowledge}

Tracking adversary knowledge is a central tool for the applications to
follow. We describe this core concept adopting the notation and
conventions of Clarkson's experimental protocol
\cite{clarkson09quantifying}.

Bob seeks the answer of a query, a function evaluated on Alice's data.
The data here is secret in that Bob does not already know it.
Alice wants to protect her secret, making sure that answering
Bob's query is safe.
To do this, Alice will explictly track what Bob
learns about her secret, treating Bob as a rational adversary
who will use any information provided to him in order to learn Alice's
secret. Specifying the protection of the secret will be addressed shortly.

Alice's secret data is a value $ \rsecret $ which Bob knows is in some
set of possibilities $ \secrets $. Furthermore, Bob considers some
values more likely than others, a probability distribution $ \delta_1
: \secrets \ra \Real $. We will refer to this distribution as Bob's
\emph{belief}. Alice then considers the effect of letting Bob learn
the output of a potentially non-deterministic query function $ Q $, written
in a probabilistic language. She can do this by computing the
conditional distribution $ \delta_2 \defeq \delta_1 \cond (Q(\rsecret)
= o) $ for some potential output value $ o $. Assuming $ \delta | C $
indeed implements \emph{exact} probabilistic conditioning, this
distribution $ \delta_2 $ will be the belief Bob would attain, were he
to see that the function $ Q $ outputs $ o $ given input $ \rsecret $.

If Bob indeed does learn this output, Alice can revise her
representation of Bob's belief to $ \delta_2 $ which she can then
revise further if Bob learns additional functions over her data. In
this manner Alice explicitly tracks what Bob knows about her secret
as a probability distribution. The problems of how Alice knew $
\delta_1 $, Bob's initial belief, in the first place and whether she
can actually perform exact inference will need to be
addressed.

\paragraph*{Knowledge-based policy enforcement}

Alice can use the tracked belief in a variety of ways, the most direct
of which is to judge the information security or safety of $ Q $ in
order to determine whether she should let Bob get access to the output
of the function. To do this, Alice starts by defining a
knowledge-based policy $ P : \dists \ra \set{\strue,\sfalse} $, where
$ \dists $ is a set of all distributions over $ \secrets $. For
example, Alice might require that Bob never becomes too certain of her
value, so she defines $ P(\delta) \defeq \forall \asecret \in \secrets \qsep
\delta(\asecret) \leq t $ for some threshold $ t $. She will then deem the
query $ Q $ safe whenever $ P(\delta_2) $ holds. This particular
policy is an application of Smith's vulnerability metric
\cite{smith09foundations}, which measures the expected likelihood of
Bob guessing a secret value correctly in one try.

Unfortunately for Alice, exact probabilistic conditioning is expensive
so she decides has to use an approximation. Instead of computing $
\delta_2 $, she gets $ \delta_2^\sim \defeq \delta_1 \acond
\paren{Q(\rsecret) = o} \approx \delta_1 | \paren{Q(\rsecret) = o} $. The
approximation $ \delta_2^\sim $ here need not be a proper probability
distribution. She does not want the approximate inference to conclude
a query is safe when it really is not so she requires the
approximation to be sound relative to her policy:

\begin{definition-un} A probabilistic inference method $ |^\sim $ is
  \emph{sound relative to a policy $ P $} iff for every belief $
  \delta $, if $ P(\delta | C) $ fails then $ P(\delta |^\sim C) $
  fails.
\end{definition-un}

For Alice's example policy, a sound inference method would have to
never underestimate the probability of any secret, though it could
over-approximate it. In recent work \cite{mardziel11belief}, we
demonstrate a probabilistic language capable of this approximate, but
sound inference. Given the definition of conditional probability, $
Pr(A | B) = Pr(A \wedge B) / Pr(B) $, the probabilistic interpreter
maintains an over-approximation of $ Pr(A \wedge B) $ and an
under-approximation of $ Pr(B) $, leading to a sound upper bound for $
Pr(A | B) $. The work shows how Alice can enforce her policy
without revealing anything about her secret to Bob. Further application
of these ideas were proposed to protect information of multiple
mutually distrusting parties participating in secure multi-party
computation \cite{mardziel12smc}.

%Before allowing Bob to observe the output of the
%program $ Q $, she can make sure he will attain a safe level of
%knowledge, according to policy $ P $. If $ Q $ is deemed unsafe, she
%could either reject it or modify $ Q $ (obfuscate, add noise, etc.) to
%make it more safe. Either way, Alice has to be careful in how she
%performs these tasks. 
%
%\pxm{If there is no space, cut all the safe policy enforcement stuff
%  below.}
%
%\paragraph*{Safe policy enforcement} The way Alice enforces her policy
%$ P $ could leak information about her secret if she evaluates this policy on
%Bob's conditioned knowledge, conditioned on output of $ Q $, given her
%actual secret input $ s $. That is, $ \delta_2^\sim $ depends on $ s
%$; if Bob learns that $ Q $ is unsafe, $ P(\delta_2^\sim) = \sfalse $,
%he might learn something about $ s $.

%$$ P^s(\delta) \defeq \forall s, \forall o \;\; P(\delta \acond
%\paren{Q(s) = o}) = \strue $$
%
%$ P^s(\delta) = \strue $ implies $ P(\delta \acond \paren{Q(s) = o}) =
%\strue $.

If Alice deems $ Q $ safe, she can allow Bob access to $ Q(\rsecret) $
and update her representation of Bob's belief to $ \delta_2^\sim $. If
she rejects $ Q $, she will not allow access to $ Q(\rsecret) $ and
keep $ \delta_1 $ as the representation of Bob's unchanged
belief. Whether Alice deems $ Q $ safe or not, she can repeat the same
process for another program that might come around, starting from
either $ \delta_1 $ or $ \delta_2^\sim $, thereby maintaining an
explicit representation of Bob's knowledge while enforcing her
knowledge-based policy.

\paragraph*{Obfuscation/Noising} Probabilistic computation is also very
convenient for discerning the effects of obfuscation or noising
mechanisms, via simple composition with the query function $ Q $.

If Alice does not want Bob to learn the output of $ Q $ directly, she
can add obfuscation to its input, or noising to its output. Let $ O $
be a potentially non-deterministic obfuscation function, written in a
probabilistic language, that takes in a value $ \asecret \in \secrets
$ and produces another $ \asecret' \in \secrets $. Depending on the
structure of $ \secrets $, this obfuscation might be in the form of
resolution reduction, permutation of locations, or a variety of other
mechanisms. Alice can then consider Bob's knowledge were he to learn
the output of an obfuscated function $ Q(O(\rsecret)) $ instead of $
Q(\rsecret) $ directly, using conditioning, $ \delta_2 = \delta_1
\cond \paren{Q(O(\rsecret)) = o} $.

Alice can also consider adding noise to the output of $ Q $, using a
noising function $ N $, again written in a probabilistic language. She
can then consider Bob's revised belief $ \delta_2 = \delta_1 \cond
\paren{N(Q(\rsecret)) = o} $. She might consider obfuscation and noise
at the same time, revising Bob's belief to $ \delta_2 = \delta_1 \cond
\paren{N(Q(O(\rsecret))) = o} $. Bob's presumed revised belief assumes
that he knows $ O $ and $ N $.

If security is the only issue then a trivial noising function, $
N(\rsecret) \defeq 42 $, ensures that Bob learns absolutely nothing
about $ \rsecret $.  This ignores utility concerns that Bob (and
Alice) might be interested in. One approach is to specify Bob's
utility in terms of the accuracy or precision of the query answer $
Q(\rsecret) $.  Depending on the structure of $ Q $, it is certainly
possible to reveal some obfuscated or noised version $Q'$ that
simultaneously satisfy both security and utility objectives.

\paragraph*{Secrets as Probablistic Programs} Likewise it might
appropriate for Alice to express her security concerns in terms of a
function of her secret data, instead of the secret data itself.
Her data might itself not be all that damaging, and in that
setting it does not make sense for Alice to protect its reconstruction
by Bob. Instead Alice can define a ``blacklist'' function $ B $
specified as a probablistic program that operates on her data, and
whose output she would like to protect from inference by Bob.

In order to check security of $ B $, and utility of $ Q $, relative
to Bob's knowledge of her secret, she can evaluate these functions on
the probabilistic value representing Bob's knowledge, written $
B(\delta) $ and $ Q(\delta) $, respectively. Alice can then attempt to
add obfuscation/noise while making sure some agree-upon utility policy
$ U $ will be satisfied on Bob's view on the output of $ Q $.

\begin{definition-un}
Given belief $ \delta $, security policy $ P $ and utility policy $ U
$, are satisfied for blacklist function $ B $ and utility function $ Q
$ whenever $P(B(\delta))$ and $U(W(\delta))$ hold. 
\end{definition-un}

Alice can determine the safety of $ B(\delta) $ and utility of $
Q(\delta) $ in a variety of ways. She could use the policy $ P $ from
earlier on $ \delta_B $, bounding Bob's chances of guessing the output
of $ B $. She can also use the negation of $ P $, specifically $
U(\delta) \defeq \neg P(\delta) = \exists o \qsep \delta(o) > t $, to
measure some minimum level of utility from $ Q $, by checking that $
U(Q(\delta)) $ holds. Alternate means could involve the ubiquitous
entropy quantity to measure a lower bound on the entropy of $ B(\delta)
$ and an upper bound on the entropy of $ Q(\delta) $.

%For example, Alice might not
%want Bob to determine her religion from her secret information, but
%agrees that it is acceptable that Bob determines her age, regardless
%of whether these properties are directly part of the fields of Alice's
%secret $ \rsecret $.

Having the ability to measure the secrecy and utility of Bob's belief,
Alice can then determine how to construct a function $ Q'(\asecret) =
N(Q(O(\asecret))) $ to ensure $ \delta \cond \paren{Q'(\rsecret) = o}
$ satisfies the security and utility policies (when possible).  In
recent work \cite{chakraborty12balancing} we demonstrate how certain
sets of features of $ \asecret $ can be safely shared, while
preventing Bob from guessing the output of a blacklist function $ B $
defined on same set of features.

\pxm{I don't understand the below paragraph.}  Further applications of
these ideas in the framework of belief tracking presented here remains
to be explored. We have not considered yet designing obfuscating and
noising for arbitrary query functions and blacklist functions.  If we
assume obfuscation and noising functions are specified as
probabilistic programs with free parameters, a naive method to satisfy
security and utility policies could be to enumerate and choose among
the choices for these free parameters.  An open question is to ask
whether the specification of obfuscation and noising functions can be
posed as a problem of inference, can we efficiently and mechanically
\emph{learn} appropriate choices for them?  

\paragraph*{Predictive models for blacklist functions} The blacklist
(and utility) functions are useful to protect information which is not
part of the secret value $ \rsecret $, but is somehow correlated with
the secret value. For example, Alice's simple demographical
information like age, postal code, and gender might compose the
structure of $ \rsecret $ but Alice might be more interested in
protecting her religions preference, even though this is not a field
in the system she is securing.

Predictive models or classifiers, however, might exist that do a
reasonable job of predicting religion given demographical
information. Naturally, Alice could specify one such model as her
blacklist function $ B $. Unfortunately this does not stop Bob from
using a different model $ B $ which is as good or better at predicting
Alice's religion. This is especially problematic when there are two
disjoint sets of fields in Alice's secret $ \rsecret $ that are both
good for predicting religion. Alice might obfuscate one, but leave the
second set completely untouched.

In general it is impossible for Alice to ensure \emph{all} possible
models for religion are defeated by her noising/obfuscation
functions. This is simply due to Bob's potential model $ B(s) = r $
where $ r $ is a constant that just happens to be Alice's religion. To
make this problem reasonable, we restrict ourselves to models that
Bob, assumed rational, could produce, given some public body of data
available to both Alice and Bob.

\paragraph*{Background knowledge and future work} An assumption on the common
background data is useful for both limiting the sorts of blacklist
functions $ B $ Alice should be concerned about and for determining
what Bob's initial belief $ \delta $ about Alice's secret fields can
be.

If we assume Bob takes advantage of a common labeled dataset $ D_r $
of tuples $ (s_i, r_i) $ tying demographical information to religion,
Bob can learn a variety of predictive models for religion via
supervised machine learning techniques. Alice would like to therefore
protect herself (via blacklist) against any model that from Bob's
point of view is accurate (in that it does well on predicting religion
in $ D_r $). How to do this effectively is a challenging aspect we
would like to address in the future.

A related problem is how Alice can construct $ \delta $, the initial
belief Bob has about her secret value. The use of probability
distributions to model Bob's knowledge already presumes that Alice's
secret value $ \rsecret $ is a sample from $ \delta $. It is natural,
then, to presume a shared data set $ D $ instances $ \asecret_i $ among the
population Alice belongs to, is also composed of samples from that
distribution.

Alice could thus assume Bob would form an initial belief about her via
a statistical model with some unknown parameters $ p $ with an initial
distribution on $ p $, labeled $ \delta_p $ and sampling function $ S
$, written in a probabilistic language, that produces samples $
\asecret \in \secrets $ given parameters $ p $. Alice could then infer
these parameters by conditioning on the samples from this distribution
being $ D $, specifically $ \delta_p \cond (S(p) = \asecret_1) \cond
(S(p) = \asecret_2) \cond \cdots $ as is demonstrated in recent work
of Gordon et al.\cite{gordon13model}. Bob's initial belief about $
\rsecret $ would then be the distribution $ S(\delta_p) $.

The problem here is the same as in the problem of specifying blacklist
functions; Alice does not know what model Bob could use, either to
infer her religion, or to form an initial belief. Alice could take
into account multiple models, thereby having to track several options
for Bob's belief, but how to construct a sufficient set of models is
another challenging element of our future work.

%Our goal is then, given a set of data instances $ D $, with two class
%attributes, one ``whitelist'', and one ``blacklist'', to create
%noising/obfuscation functions that, when applied to the data
%instances, will make \emph{all} classifiers for the blacklist
%attribute fail (perform poorly) but make some classifiers for the
%whitelist attribute perform well. \pxm{todo: Write this in a nicer
%  way. Thinking about this now, it seems like it can all be described
%  without need for probabilistic computing.}

%From a set of blacklist functions $\blacklists$ and a set of whitelist
%functions $\whitelists$ each a predictive model specified as a
%probabilistic program, our goal is to then create noising/obfuscation
%functions, that $ \forall B \in \blacklists $ satisfies the security
%policy $ P(B(\delta_2^\sim)) $, while $ \exists W \in \whitelists $
%satisfies the utility policy $ U(W(\delta_2^\sim)) $.  In
%\cite{chakraborty12balancing} we demonstrated one means of learning
%obfuscation functions that satisfy security and utility policies in
%sharing with Bob an obfuscated version of Alice's secret value, $ Q =
%O(s) $, via suppressing and selecting to share only particular
%features of Alice's secret information.

% \subsubsection*{Acknowledgments}

%\cite{mardziel12smc} \cite{mardziel11belief}
%\cite{smith09foundations} \cite{clarkson09quantifying}
%\cite{alvim12gain} \cite{yasuoka10quantitative}
%\cite{gordon13model} \cite{kenthapadi05simulatable}
%\cite{rastogi09relationship} \cite{yasuoka11bounding}

\bibliographystyle{plain}
\bibliography{ppw}

\end{document}


%\begin{abstract}
%\end{abstract}

% \section{Introduction}

%\begin{figure}[h]
%\begin{center}
%%\framebox[4.0in]{$\;$}
%\fbox{\rule[-.5cm]{0cm}{4cm} \rule[-.5cm]{4cm}{0cm}}
%\end{center}
%\caption{Sample figure caption.}
%\end{figure}

% \subsection{Background}

%\begin{table}[t]
%\caption{Sample table title}
%\label{sample-table}
%\begin{center}
%\begin{tabular}{ll}
%\multicolumn{1}{c}{\bf PART}  &\multicolumn{1}{c}{\bf DESCRIPTION}
%\\ \hline \\
%Dendrite         &Input terminal \\
%Axon             &Output terminal \\
%Soma             &Cell body (contains cell nucleus) \\
%\end{tabular}
%\end{center}
%\end{table}
